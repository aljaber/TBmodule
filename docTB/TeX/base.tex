\documentclass[12pt,a4paper,italian]{article}


\usepackage[italian]{babel}
\usepackage[latin1]{inputenc}
\usepackage{amsmath}
\usepackage{amsfonts}
\usepackage{amssymb}
\usepackage{color}
\usepackage{xcolor}
\usepackage{hyperref}
\usepackage[all]{hypcap}
\usepackage{ifthen}
\usepackage{wrapfig}
%\author{Piero Bizzotto}
\usepackage[top=2cm,bottom=5cm,left=80pt,right=80pt]{geometry}
\usepackage{graphicx}
\DeclareGraphicsExtensions{.jpg,.png}

\newcommand{\ajax}{GESTIONE DELLA COMUNICAZIONE TRA CLIENTI E OPERATORI COMMERCIALI IN UN SISTEMA CRM}
\newcommand{\sito}{\href{http://www.tecnobit.info}{http://www.tecnobit.info}}

\setlength{\parindent}{0pt} %settato indentazione di default 
\setlength{\headheight}{3cm} %settato grandezza header...in altre parole, quanto distanzio il doc dall'intestazione

\usepackage{fancyhdr} %pacchetto per le intestazioni
\pagestyle{fancy} %uso del pacchetto


\fancyhead{} %annulla head di default
\fancyfoot{} %annulla foot di default


\usepackage{lastpage} %setto pg di pgtot a rfoot
     \rfoot{pagina \thepage\ di \pageref{LastPage}}


\lfoot{Versione: \insertversion} %setto versione doc a lfoot
\renewcommand{\footrulewidth}{0.5pt} %ridefinisco il valore della riga di intestazione
\renewcommand{\headrulewidth}{0.5pt} %ridefinisco il valore della riga di pie' di pagina

\newcommand{\insertversion}{0.0} %definisco il nuovo comando per inserire la versione


%\lhead{  \begin{Huge} \ajax \end{Huge} \\  %intestazione di sinistra
				%	%\begin{Large}	Software per il Disegno Grafico\\ in Tecnologie Web \end{Large}  
	%				\begin{normalsize}\sito \end{normalsize}
	%		%\\ versione documento: \insertversion\ del \today} %setto l'intestazione sx
	%	}
\chead{ %includo logo nell'intestazione dx
	 	\includegraphics[scale=0.3]{../tecnobit.jpg}  
}


%CREAZIONE ELENCHI NUMERATI PERSONALIZZATI
\newcounter{Lcount}
\newcounter{Rcount}
\setcounter{Lcount}{0}
\setcounter{Rcount}{0}

\newenvironment{elenconumerato}[2][ ]
{
  \begin{list}{#1\arabic{Lcount}.}
    {
	\setcounter{Rcount}{\value{Lcount}}
	\setcounter{Lcount}{0} 
	\usecounter{Lcount} 
\addtolength{\leftmargin}{#2pt}
	}
}
{
  \end{list}
 \setcounter{Lcount}{\value{Rcount}}
}

%CREAZIONE ELENCHI PUNTATI
\newenvironment{elencopuntato}[1][]
{
\begin{list}{\textbullet} %\itemindent=#1pt
	{
	\addtolength{\leftmargin}{#1pt}
	}
} 
{
\end{list}
}


\newenvironment{elencodescrittivo}[1][]{\begin{description} \setlength{\itemindent}{#1pt} \addtolength{\leftmargin}{#1pt}} {\end{description}}

\newcommand{\TITOLODOC}{Titolo}

%footer centrale
\cfoot{ \TITOLODOC \\  E-mail:    \href{ mailto:piero.bizzotto@gmail.com}{ piero.bizzotto@gmail.com}  }

%INSERIMENTO IMMAGINI
\newcommand{\imagerealsize}[1]{\vspace{20pt} \includegraphics{#1} }
\newcommand{\imageadapted}[1]{\vspace{20pt} \includegraphics[width=1\textwidth]{#1} }

\newcommand{\glosspath}{.\glossario}
\newcommand{\gloss}[1]{\hyperref{\glosspath~\glossario.pdf}{}{#1}{#1}}

\hypersetup{
    %bookmarks=true,         % show bookmarks bar?
    %unicode=false,          % non-Latin characters in Acrobat’s bookmarks
	%pdftoolbar=true,        % show Acrobat’s toolbar?
	%pdfmenubar=true,        % show Acrobat’s menu?
    %pdffitwindow=true,      % page fit to window when opened
    %pdftitle={My title},    % title
    %pdfauthor={Author},     % author
    %pdfsubject={Subject},   % subject of the document
    %pdfnewwindow=true,      % links in new window
    %pdfkeywords={keywords}, % list of keywords
    colorlinks=true,         % false: boxed links; true: colored links
    linkcolor=black,           % color of internal links
    %citecolor=green,        % color of links to bibliography
    %filecolor=magenta,      % color of file links
    urlcolor=teal    % color of external links
%	linktocpage=false;
}


%COLORAZIONE TESTO
\newcommand{\blue}[1]{{\color {blue} #1}} 
\newcommand{\red}[1]{{\color {red} #1}}
\newcommand{\green}[1]{{\color {green} #1}}
\newcommand{\sezione}[1]{\leftskip=0pt \section{#1} \leftskip=18pt}
\newcommand{\subsezione}[1]{\leftskip=18pt \subsection{#1} \leftskip=36pt}
\newcommand{\subsubsezione}[1]{\leftskip=36pt \subsubsection{#1} \leftskip=54pt}
\newcommand{\subsubsecindent}{54}
\newcommand{\subsecindent}{36}
\newcommand{\secindent}{18}
\newcommand{\normindent}{8}
\newcommand{\code}[1]{{\bfseries \texttt{#1}}}
\newcommand{\paragrafo}[1]{\leftskip=36pt \paragraph{#1} \leftskip=54pt}
\newcommand{\subparagrafo}[1]{\leftskip=54pt \subparagraph{#1} \leftskip=72pt}